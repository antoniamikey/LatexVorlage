% !TEX encoding = UTF-8 Unicode
% !TEX root =  Main.tex
% Eigene Befehle und typographische Auszeichnungen für diese

%%%%%%%%%%%%%%%%%%%%%%%%%%%%%%%%%%%%%%%%%%%%%%%%%%%%%%%%%%%%%%%%%%%%%%%
% SCHRIFT UND TYPOGRAPHIE
%%%%%%%%%%%%%%%%%%%%%%%%%%%%%%%%%%%%%%%%%%%%%%%%%%%%%%%%%%%%%%%%%%%%%%%
% Schriftart wechseln
\newcommand{\changefont}[3]{\fontfamily{#1} \fontseries{#2} \fontshape{#3} \selectfont}


%%%%%%%%%%%%%%%%%%%%%%%%%%%%%%%%%%%%%%%%%%%%%%%%%%%%%%%%%%%%%%%%%%%%%%%
% ABKÜRZUNGEN, ZITATE UND REFERENZEN
%%%%%%%%%%%%%%%%%%%%%%%%%%%%%%%%%%%%%%%%%%%%%%%%%%%%%%%%%%%%%%%%%%%%%%%
% Abkürzungen mit korrektem Leerraum
\newcommand{\ua}{\mbox{u.\,a.\ }}
\newcommand{\zB}{\mbox{z.\,B.\ }}
\newcommand{\dahe}{\mbox{d.\,h.\ }}
\newcommand{\Vgl}{Vgl.\ }
\newcommand{\bzw}{bzw.\ }
\newcommand{\evtl}{evtl.\ }

% Hilfsbefehle für Zitate
% \newcommand{\maybeAuthor}[1]{%
%   \ifblank{\citeauthor{#1}}{o.\,A.}{\citeauthor{#1}}%
% }
% \newcommand{\maybeYear}[1]{%
%   \ifblank{\citeyear{#1}}{o.\,J.}{\citeyear{#1}}%
% }
% Autor mit Fallback auf o. A.

% Zitierformate
% Damit das Komma nicht verschluckt wird:
\newcommand{\Zitat}[2]{%
  \ifstrempty{#2}
    {\parencite{#1}}%
    {\begingroup
      \def\postnotedelim{,\space}%
      \parencite[][S.~#2]{#1}%
     \endgroup}%
}

\newcommand{\VglZitat}[2]{%
  \ifstrempty{#2}
    {\parencite[Vgl.\ ][]{#1}}%
    {\begingroup
      \def\postnotedelim{,\space}%
      \parencite[Vgl.\ ][S.~#2]{#1}%
     \endgroup}%
}

\newcommand{\NarrativZitat}[2]{%
  \citeauthor{#1} (\citeyear{#1}%
  \ifstrempty{#2}{}{, S.~#2})%
}
\newcommand{\AutorName}[1]{\textsc{#1}}
\newcommand{\Autor}[1]{\AutorName{\citeauthor{#1}}}

% Abbildungsreferenzen
\newcommand{\Abbildung}[1]{Abbildung~\ref{fig:#1}}
\newcommand{\bs}{$\backslash$}

% verschiedene Befehle um Wörter semantisch auszuzeichnen 
\newcommand{\Fachbegriff}[2][\empty]{\ifthenelse{\equal{#1}{\empty}}{\textit{#2}\xspace}{\textit{#2}\xspace\footnote{#1}\nomenclature{#2}{#1}}}
\newcommand{\FachbegriffSpezialA}[4]{\textit{#4}\footnote{#3}\label{fn:#1}\nomenclature{#4}{#2. Siehe auch Fu{\ss}zeile auf Seite~\pageref{fn:#1}.}}
\newcommand{\FachbegriffSpezialB}[5]{\textit{#5}\footnote{#3}\label{fn:#1}\nomenclature{#4}{#2. Siehe auch Fu{\ss}zeile auf Seite~\pageref{fn:#1}.}}

%%%%%%%%%%%%%%%%%%%%%%%%%%%%%%%%%%%%%%%%%%%%%%%%%%%%%%%%%%%%%%%%%%%%%%%
% WÄHRUNGEN UND BETRÄGE
%%%%%%%%%%%%%%%%%%%%%%%%%%%%%%%%%%%%%%%%%%%%%%%%%%%%%%%%%%%%%%%%%%%%%%%
% Beträge mit verschiedenen Währungssymbolen
\newcommand{\Betrag}[2][general]{#2\,\ifthenelse{\equal{#1}{dollar}}{\$}{}\ifthenelse{\equal{#1}{euro}}{€}{}\ifthenelse{\equal{#1}{yen}}{¥}{}\ifthenelse{\equal{#1}{cent}}{¢}{}\ifthenelse{\equal{#1}{pound}}{£}{}\ifthenelse{\equal{#1}{peso}}{₱}{}\ifthenelse{\equal{#1}{baht}}{฿}{}\ifthenelse{\equal{#1}{franc}}{₣}{}\ifthenelse{\equal{#1}{lira}}{₤}{}\ifthenelse{\equal{#1}{drachma}}{₯}{}\ifthenelse{\equal{#1}{pfennig}}{₰}{}\ifthenelse{\equal{#1}{general}}{¤}{}}

%%%%%%%%%%%%%%%%%%%%%%%%%%%%%%%%%%%%%%%%%%%%%%%%%%%%%%%%%%%%%%%%%%%%%%%
% CODE UND TECHNISCHE AUSZEICHNUNGEN
%%%%%%%%%%%%%%%%%%%%%%%%%%%%%%%%%%%%%%%%%%%%%%%%%%%%%%%%%%%%%%%%%%%%%%%
% Technische Elemente
\newcommand{\Eingabe}[1]{\texttt{#1}}
\newcommand{\Code}[1]{\texttt{#1}}
\newcommand{\Datei}[1]{\texttt{#1}}
\newcommand{\Datentyp}[1]{\textsf{#1}}
\newcommand{\XMLElement}[1]{\textsf{#1}}
\newcommand{\Webservice}[1]{\textsf{#1}}

%%%%%%%%%%%%%%%%%%%%%%%%%%%%%%%%%%%%%%%%%%%%%%%%%%%%%%%%%%%%%%%%%%%%%%%
% TABELLEN UND ABBILDUNGEN
%%%%%%%%%%%%%%%%%%%%%%%%%%%%%%%%%%%%%%%%%%%%%%%%%%%%%%%%%%%%%%%%%%%%%%%
% Beschriftungen anpassen
\addto\captionsngerman{
	\renewcommand{\figurename}{Abb.}
	\renewcommand{\tablename}{Tab.}
}

% Spezielle Spaltentypen
\newcolumntype{x}[1]{>{\raggedleft\hspace{0pt}}p{#1}}
\newcolumntype{y}[1]{>{\RaggedRight\arraybackslash\hsize=#1\hsize}X}
