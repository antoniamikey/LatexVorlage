% !TEX encoding = UTF-8 Unicode
% !TEX root =  Main.tex

% Meta-Informationen ------------------------------------------------------------------------------------
%   Definition von globalen Parametern, die im gesamten Dokument verwendet
%   werden können (z.B auf dem Deckblatt etc.).
%
%   ACHTUNG: Wenn die Texte Umlaute oder ein Esszet enthalten, muss der folgende
%            Befehl bereits an dieser Stelle aktiviert werden:
%            \usepackage[latin1]{inputenc}
% -------------------------------------------------------------------------------------------------------
\newcommand{\titel}{Konzeptionierung und Entwicklung einer Softwarelösung für einen automatisierten und
datenschutzkonformen Bearbeitungsprozess von Anträgen am Beispiel des AStA der HWR Berlin}
\newcommand{\untertitel}{}
\newcommand{\untertitelDeckblatt}{}
\newcommand{\art}{Hausarbeit}
\newcommand{\fachgebiet}{}
\newcommand{\autor}{Antonia Schikora}
\newcommand{\keywords}{Hausarbeit, Antonia Schikora}
\newcommand{\fachbereich}{}
\newcommand{\fachrichtung}{Wirtschaftsinformatik\xspace}
\newcommand{\studienjahrgang}{WI23A}
\newcommand{\semester}{4. Semester}
\newcommand{\matrikelnr}{tbd}
\newcommand{\erstgutachter}{Schlesi}
\newcommand{\zweitgutachter}{Denny}
\newcommand{\hochschule}{Hochschule für Wirtschaft und Recht Berlin}
\newcommand{\ort}{Berlin}
\newcommand{\ausbildungsbetrieb}{tbd}
\newcommand{\logo}{Logo_HWR.png}
\newcommand{\creator}{LaTeX}
\newcommand{\datum}{\today}  % Fügt das aktuelle Datum ein

