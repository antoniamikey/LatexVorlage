% !TEX encoding = UTF-8 Unicode
% !TEX root =  ../Main.tex

\chapter{Einleitung}
\label{cha:Einleitung}


\section{Das Ziel dieser Arbeit}
\label{sec:ZielDerArbeit}
das ist das Ziel 
djufhydkjfhydiofgjydfgomökhhj \gls{rup}

\section{Die Umgebung, in der die Arbeit entstand}
\label{sec:EntstehungsUmgebungArbeit}

Das ist die Umgebung der Arbeit

\section{Der Aufbau dieser Arbeit}
\label{sec:AufbauDieserArbeit}



\begin{description}

	\item[Aktueller Wissensstand:] Der aktuelle Wissensstand beschreibt, auf welchem Wissensniveau sich der Autor im Moment der Aufnahme der Arbeit befand.
	
	\item[Entwicklungsstand TYPO3:] Dieses Kapitel befasst sich mit dem grunds\"atzlichen Entwicklungsstand von TYPO3 Version 4 und 5 und den mit der Extension-Entwicklung zusammenh\"angenden Frameworks Extbase, Fluid und FLOW3. Es werden grundlegende Eigenschaften der Frameworks und deren Leistungsf\"ahigkeit skizziert.
	
	\item[Die Planung des Webradio-Players:] Dieses Kapitel umfasst die Dokumentation der gesamten Planungsphase des Webradio-Players. Hier wird eine Übersicht über die bereits vorhandene Lösung geschaffen und anschließend die zur Planung erforderlichen Dokumente des RUP angefertigt.
	
	\item[Die Entwicklung des Webradio-Players:] Dieses Kapitel enthält die Dokumentation der tatsächlichen Programmierung der Software. Hier werden die Voraussetzungen zur Implementation geklärt und der Verlauf der Entwicklung anhand von Beispielen schrittweise abgearbeitet.
	
	\item[Fazit und kritische Bewertung:] Im Fazit werden die gemachten Erfahrungen und die Ergebnisse der Planung und Entwicklung abschließend zusammengefasst und kritisch bewertet. Zusätzlich wird ein kleiner Ausblick auf Erweiterungsmöglichkeiten und mögliche Optimierungsschritte unternommen.

\end{description}


\section{Wenige Informationen, wenige Quellen \dots}
\label{sec:Quellenlage}
das ist was anders