% !TEX encoding = UTF-8 Unicode
% !TEX root =  ../Main.tex

\chapter{Projektidee}
\label{cha:Projektidee}

\section{Platzhalter}
\label{sec:tbd}

Ablauf eines Antrags:

Studierende Person
- UI wo man seine Daten eintragen kann und den Antragsgrund auswählt
- je nach Antragsgrund wird angezeigt, welche Dokumente hochgeladen werden müssen 
- Bestätigung von AGBs??? und Bestätigung dass alles rechtsmäßig ist???? 
- Antrag absenden

AStA Sozialbüro: 
- eigenes Interface, wo aktuelle Anträge angezeigt werden (wie soll das aussehen???)
- jeder neue Antrag bekommt den Tag "unbearbeitet"
- Antrag kann geöffnet werden und der/die Mitarbeiter/in kann den Antrag prüfen -> alle Felder sind beschreibar um im Nachhinein Daten ändern zu können -> Button Daten ändern


Genereller Aufbau des Projektes: 
- Server, auf dem alle Daten gespeichert werden -> der muss an die AStA Website angebunden sein (Funktioniert das mit Wordpress und wenn ja wie?)
- eigenes Interface für Mitarbeitende des AStA (Kann das auch eine Wordpress Seite sein? oder sollte das was anderes sein?) -> hosten dieser Seite auf dem eigenen Server
- Wie funktioniert die Verknüpfung hier mit Ultradox? Kann ich Ultradox auf den Server zugreifen lassen? 
- Wie speicher ich die Daten überhaupt in der Datenbank? 
- Wie kann ich die Daten änderbar machen in dem Mitarbeiterinterface? und wie kann ich das machen, dass ich dort Dokumente hochlade? 

Next Steps: 
- Recherche zu Wordpress und möglicher Anbindung an eine Datenbank 
- Aufbau einer Datenbank mit Testdaten
- Bau eines Prototyps der Website um die Funktionalität zu testen 