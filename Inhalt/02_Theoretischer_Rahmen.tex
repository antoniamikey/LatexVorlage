% !TEX encoding = UTF-8 Unicode
% !TEX root =  ../Main.tex

\glsunset{dsgvozitat}
\glsunset{aeuv}


\chapter{Theoretischer Rahmen}
\label{cha:Theoretischer_Rahmen}

\section{Anforderungen an Datenschutz und Datensicherheit in Deutschland}
\label{sec:Anforderunge}

Deutschlands regulatorische Landschaft ist ein komplexes Gebilde, das sich von europäischen Normen bis hin zu branchenspezifischen Verordnungen und technischen Richtlinien erstreckt. Im Hinblick auf die datenschutzbezogenen Gesetzgebungen befindet sich auf der ersten Ebene der Pyramide die „Verordnung (EU) 2016/679 des Europäischen Parlaments und des Rates vom 27. April 2016 zum Schutz natürlicher Personen bei der Verarbeitung personenbezogener Daten, zum freien Datenverkehr und zur Aufhebung der Richtlinie 95/46/EG (Datenschutz-Grundverordnung)“. [\gls{dsgvozitat}]

Die DSGVO bestimmt Grundsätze und Vorschriften, um natürliche Personen bei der Verarbeitung ihrer personenbezogenen Daten zu schützen. Dadurch soll die Einhaltung ihrer Grundrechte sowie Grundfreiheiten gewährleistet werden. [Vgl. §1 \gls{dsgvozitat}] Dabei sind Daten gemeint, die – egal, ob es sich um eine automatisierte Verarbeitung handelt oder nicht – in einem Dateisystem gespeichert sind oder gespeichert werden sollen. [Vgl. §2 \gls{dsgvozitat}] 
Personenbezogene Daten meint dabei alle Informationen, die einer natürlichen Person zuzuordnen sind. [Vgl. §4 \gls{dsgvozitat}]

Die Grundsätze der \gls{dsgvo} besagen, dass die Daten nur auf eine rechtmäßige und für die betroffene Person nachvollziehbare Weise verarbeitet werden dürfen. Dabei muss darauf geachtet werden, dass nur die maximal notwendige Anzahl an Daten ausschließlich für den festgelegten Zweck verwendet wird. Die Daten müssen sachlich korrekt sein und dürfen nur für einen bestimmten Zeitraum vom Verwendenden gespeichert werden. Die Verarbeitung muss durch technische und organisatorische Maßnahmen geschützt sein. Der/die Verantwortliche muss dafür sorgen, dass diese Grundsätze eingehalten werden und diese Einhaltung auch nachweisen kann. [Vgl. §5 \gls{dsgvozitat}]

Eine rechtmäßige Verarbeitung besteht beispielsweise dann, wenn die Person, deren personenbezogene Daten verarbeitet werden, dieser Verarbeitung zugestimmt hat. [Vgl. §6 Abs. 1 lit. a \gls{dsgvozitat}] Die/der Verantwortliche muss eine solche Einwilligung auch nachweisen können. Die Einwilligung kann zudem jederzeit widerrufen werden. [Vgl. §7 \gls{dsgvozitat}] Ein weiteres Szenario ist, wenn die Verarbeitung der Daten zur Erfüllung eines Vertrags notwendig ist, dessen Vertragspartei die betroffene Person ist. [Vgl. §6 Abs. 1 lit. b \gls{dsgvozitat}] Weitere rechtsmäßige Szenarien sind in §6 \gls{dsgvozitat} beschrieben.  

Sobald die personenbezogenen Daten erhoben werden, ist die/der Verantwortliche dazu verpflichtet, der betroffenen Person den Namen und die Kontaktdaten der/des Verantwortlichen, den Verwendungszweck sowie die Rechtsgrundlage nach Artikel 6 zu nennen. [Vgl. §13 Abs. 1 \gls{dsgvozitat}] Der/die Verantwortliche muss zudem weitere Informationen zur Verfügung stellen, die für eine faire und transparente Verarbeitung notwendig sind. [Vgl. §13 Abs. 2 \gls{dsgvozitat}]

