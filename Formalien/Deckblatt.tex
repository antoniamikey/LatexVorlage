% !TEX encoding = UTF-8 Unicode
% !TEX root =  Main.tex

% Kopfbereich mit Logo
\titlehead{%
  % Verschiebt das Logo nach oben
  \vspace*{-1,5cm}%
  % Zentriert das Logo
  \centering%
  % Fügt das HWR-Logo ein (60% der Textbreite)
  \includegraphics[width=0.6\linewidth]{Bilder/Logo_HWR.png}%
  % Abstand nach dem Logo
  \vspace{1,5cm}%  % Weniger Abstand nach dem Logo
}

% Art der Arbeit als Überschrift
\vspace{10cm} 
\subject{\art}

% Erstellt den Titelbereich mit Rahmen
\title{%
  \vspace*{-1,5cm}%  % Verschiebt den gesamten Titelbereich nach oben
  \normalfont\endgraf
  \rule{\textwidth}{.4pt}\\\relax
  \vspace{0.5cm}%  % Kleiner Abstand nach dem Zeilenumbruch
  \begingroup
    \centering
    % Zeilenabstand auf 1.5
    \linespread{1.5}\selectfont
    % Titel in großer Schrift
    \large\titel\\
    % Untertitel in normaler Schrift (falls vorhanden)
    % \normalsize\untertitel
  \endgroup%  % Weniger Abstand vor der unteren Linie
  \endgraf\rule{\textwidth}{.4pt}%
}

% Setzt das Datum und die Hochschule
\date{\vspace{-1.5cm}%  % Reduziert den Abstand zum Titelbereich
\normalsize vorgelegt am \datum\\
an der \\ 
Hochschule für Wirtschaft und Recht Berlin\vspace{5mm}}

% Erstellt die Tabelle mit den Informationen
\publishers{%
  \begin{tabular}{l l}
    % Persönliche Informationen
    \textbf{\normalsize{Von: }} & \normalsize{\autor} \\
    \textbf{\normalsize{Fachrichtung:}} & \normalsize{\fachrichtung} \\
    \textbf{\normalsize{Studienjahrgang:}} & \normalsize{\studienjahrgang} \\
    \textbf{\normalsize{Semester:}} & \normalsize{\semester} \\
    \textbf{\normalsize{Ausbildungsbetrieb:}} & \normalsize{\ausbildungsbetrieb} \\
    % Betreuer-Informationen
    \textbf{\normalsize{Betreuender Prüfer:}} & \normalsize{\erstgutachter} \\
    \textbf{\normalsize{Betreuer Betrieb:}} & \normalsize{\zweitgutachter} \\
    % Unterschriftszeile
    \textbf{\normalsize{Unterschrift Betreuer: }} & \underline{\hspace{3cm}} \\
  \end{tabular}
}

% Keine Seitenzahl auf dem Deckblatt
\pagestyle{empty}
% Erstellt das Deckblatt
\maketitle