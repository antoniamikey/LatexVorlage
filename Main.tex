% !TEX encoding = UTF-8 Unicode
% !TEX root =  Main.tex
%------------------------------------------------------------------------------
%	Formatvorlage für wissenschaftliche Arbeiten (Diplomarbeit, Bachelorarbeit, Masterarbeit)
%------------------------------------------------------------------------------
%	Version: 2.0
%	Datum: 06.06.2025


% Dokumentenkopf ---------------------------------------------------------------------------------------
%   Diese Vorlage basiert auf "scrreprt" aus dem koma-script.
% ------------------------------------------------------------------------------------------------------
\documentclass[
    12pt, % Schriftgröße
    DIV=10, % Änderung der Größe des Satzspiegels (bedruckbarer Bereich einer Seite), nur in Verbindung mit koma-script verwendbar
    ngerman, % für Umlaute, Silbentrennung etc.
    a4paper, % Papierformat
    oneside, % einseitiges Dokument
    titlepage, % es wird eine Titelseite verwendet
    parskip=half, % Abstand zwischen Absätzen (halbe Zeile)
    headings=normal, % Größe der Überschriften verkleinern
    listof=totoc, % Verzeichnisse im Inhaltsverzeichnis aufführen
    bibliography=totoc, % Literaturverzeichnis im Inhaltsverzeichnis aufführen
    index=totoc, % Index im Inhaltsverzeichnis aufführen
    captions=tableheading, % Beschriftung von Tabellen unterhalb ausgeben
    final % Status des Dokuments (final/draft)
]{scrreprt}

% UTF8 und T1 Fontencoding -----------------------------------------------------------------------------
\usepackage[utf8]{inputenc}
\usepackage[T1]{fontenc}
\usepackage{tikz} % Ensure TikZ/PGF is loaded after fontenc/inputenc

% !TEX encoding = UTF-8 Unicode
% !TEX root =  Main.tex

% Meta-Informationen ------------------------------------------------------------------------------------
%   Definition von globalen Parametern, die im gesamten Dokument verwendet
%   werden können (z.B auf dem Deckblatt etc.).
%
%   ACHTUNG: Wenn die Texte Umlaute oder ein Esszet enthalten, muss der folgende
%            Befehl bereits an dieser Stelle aktiviert werden:
%            \usepackage[latin1]{inputenc}
% -------------------------------------------------------------------------------------------------------
\newcommand{\titel}{Konzeptionierung und Entwicklung einer Softwarelösung für einen automatisierten und
datenschutzkonformen Bearbeitungsprozess von Anträgen am Beispiel des AStA der HWR Berlin}
\newcommand{\untertitel}{}
\newcommand{\untertitelDeckblatt}{}
\newcommand{\art}{Hausarbeit}
\newcommand{\fachgebiet}{}
\newcommand{\autor}{xy}
\newcommand{\keywords}{Hausarbeit, xy}
\newcommand{\fachbereich}{}
\newcommand{\fachrichtung}{Wirtschaftsinformatik\xspace}
\newcommand{\studienjahrgang}{WI23A}
\newcommand{\semester}{4. Semester}
\newcommand{\matrikelnr}{tbd}
\newcommand{\erstgutachter}{xy}
\newcommand{\zweitgutachter}{xy}
\newcommand{\hochschule}{Hochschule für Wirtschaft und Recht Berlin}
\newcommand{\ort}{Berlin}
\newcommand{\ausbildungsbetrieb}{tbd}
\newcommand{\logo}{Logo_HWR.png}
\newcommand{\creator}{LaTeX}
\newcommand{\datum}{\today}  % Fügt das aktuelle Datum ein


% !TEX encoding = UTF-8 Unicode
% !TEX root =  Main.tex

%%%%%%%%%%%%%%%%%%%%%%%%%%%%%%%%%%%%%%%%%%%%%%%%%%%%%%%%%%%%%%%%%%%%%%%
% GRUNDLEGENDE DOKUMENTEINSTELLUNGEN
%%%%%%%%%%%%%%%%%%%%%%%%%%%%%%%%%%%%%%%%%%%%%%%%%%%%%%%%%%%%%%%%%%%%%%%
% Seitenlayout und Kopfzeilen
\usepackage[
    automark,     % Kapitelangaben in Kopfzeile automatisch erstellen
    headsepline,  % Trennlinie unter Kopfzeile
    ilines        % Trennlinie linksbündig ausrichten
]{scrlayer-scrpage}

% Seitengeometrie und Abstände
\usepackage{geometry}
\usepackage{setspace}

% Fortlaufende Nummerierung
\usepackage{chngcntr}

%%%%%%%%%%%%%%%%%%%%%%%%%%%%%%%%%%%%%%%%%%%%%%%%%%%%%%%%%%%%%%%%%%%%%%%
% SPRACHE UND SCHRIFT
%%%%%%%%%%%%%%%%%%%%%%%%%%%%%%%%%%%%%%%%%%%%%%%%%%%%%%%%%%%%%%%%%%%%%%%
% Spracheinstellungen
\usepackage[ngerman]{babel}

% Schriftart und Kodierung
\usepackage{newtxtext,newtxmath}  % Times New Roman als Hauptschrift
\renewcommand{\familydefault}{\rmdefault}
\addtokomafont{disposition}{\rmfamily}
\addtokomafont{chapter}{\fontsize{13pt}{13pt}\selectfont}
\addtokomafont{section}{\fontsize{13pt}{13pt}\selectfont}
\addtokomafont{subsection}{\fontsize{13pt}{13pt}\selectfont}

% Schriftgrößen und Textauszeichnung
\usepackage{relsize}      % Relative Schriftgrößen
\usepackage[normalem]{ulem}  % Bessere Unterstreichungen
\usepackage{textcomp}     % Zusätzliche Symbole
\usepackage{microtype}

%%%%%%%%%%%%%%%%%%%%%%%%%%%%%%%%%%%%%%%%%%%%%%%%%%%%%%%%%%%%%%%%%%%%%%%
% ABBILDUNGEN UND GRAFIKEN
%%%%%%%%%%%%%%%%%%%%%%%%%%%%%%%%%%%%%%%%%%%%%%%%%%%%%%%%%%%%%%%%%%%%%%%
% Grundlegende Grafikunterstützung
\usepackage{graphicx}
\graphicspath{{Bilder/}}

% Erweiterte Grafikfunktionen
\usepackage{svg}
\usepackage{tikz}
\usepackage[vflt]{floatflt}  % Textumfluss
\usepackage{subfigure}       % Teilabbildungen
\usepackage{rotating}        % Rotation von Elementen

%%%%%%%%%%%%%%%%%%%%%%%%%%%%%%%%%%%%%%%%%%%%%%%%%%%%%%%%%%%%%%%%%%%%%%%
% HYPERLINKS UND PDF-EINSTELLUNGEN
%%%%%%%%%%%%%%%%%%%%%%%%%%%%%%%%%%%%%%%%%%%%%%%%%%%%%%%%%%%%%%%%%%%%%%%
% URL-Formatierung
\usepackage{url}

% PDF-Optionen und Hyperlinks
\usepackage[
    bookmarks,
    bookmarksopen=true,
    colorlinks=true,
    linkcolor=black,     % Interne Links (wie im Inhaltsverzeichnis) in Schwarz
    anchorcolor=black,
    citecolor=blue,
    filecolor=magenta,
    menucolor=black,
    urlcolor=cyan,
    plainpages=false,
    pdfpagelabels,
    hypertexnames=false,
    linktocpage,
    pdftitle={\titel\untertitel},
    pdfauthor={\autor},
    pdfcreator={\creator},
    pdfsubject={\titel\untertitel},
    pdfkeywords={\keywords}
]{hyperref}

%%%%%%%%%%%%%%%%%%%%%%%%%%%%%%%%%%%%%%%%%%%%%%%%%%%%%%%%%%%%%%%%%%%%%%%
% VERZEICHNISSE UND REFERENZEN
%%%%%%%%%%%%%%%%%%%%%%%%%%%%%%%%%%%%%%%%%%%%%%%%%%%%%%%%%%%%%%%%%%%%%%%
% Glossar und Abkürzungen
\usepackage[acronym,nomain,nopostdot,nonumberlist]{glossaries}
\makeglossaries%
\setglossarystyle{list}
\renewcommand*{\glsnamefont}[1]{\textrm{#1}}

% Literaturverzeichnis
% \usepackage[square,numbers,sort&compress]{natbib}
\usepackage{csquotes} % wurde im Log empfohlen
\usepackage[
    backend=biber,
    style=authoryear,
    language=ngerman,
    backref=true
]{biblatex}
\addbibresource{Literatur/Bibliographie.bib}


% To set square brackets for parencite, use:
\renewcommand*{\mkbibparens}[1]{\mkbibbrackets{#1}}
\DeclareRobustCommand{\addcomma}{,}
\DeclareLabelalphaTemplate{
  \labelelement{
    \field[strwidth=3,strside=left]{labelname}
  }
  \labelelement{
    \field[strwidth=2,strside=right]{year}
  }
}
\DeclareFieldFormat{labelyear}{%
  \iffieldundef{year}
    {o.\,J.} % ohne Jahr
    {#1}
}
\DeclareFieldFormat{labelname}{%
  \iffieldundef{name}
    {o.\,A.} % ohne Autor
    {#1}
}
\addbibresource{Literatur/Bibliographie.bib}
% Index
\usepackage{makeidx}
\usepackage{etoolbox}

%%%%%%%%%%%%%%%%%%%%%%%%%%%%%%%%%%%%%%%%%%%%%%%%%%%%%%%%%%%%%%%%%%%%%%%
% MATHEMATIK UND SYMBOLE
%%%%%%%%%%%%%%%%%%%%%%%%%%%%%%%%%%%%%%%%%%%%%%%%%%%%%%%%%%%%%%%%%%%%%%%
\usepackage{amsmath,amsfonts}  % Mathematische Symbole
\usepackage{eurosym}           % Währungssymbole

%%%%%%%%%%%%%%%%%%%%%%%%%%%%%%%%%%%%%%%%%%%%%%%%%%%%%%%%%%%%%%%%%%%%%%%
% TABELLEN UND AUFLISTUNGEN
%%%%%%%%%%%%%%%%%%%%%%%%%%%%%%%%%%%%%%%%%%%%%%%%%%%%%%%%%%%%%%%%%%%%%%%
% Tabellenformatierung
\usepackage{tabularx}
\usepackage{multirow}
\usepackage{array}
\usepackage{longtable}
\usepackage{ltxtable}

% Listen und Aufzählungen
\usepackage{paralist}
\usepackage{enumitem}
\setlist[description]{font=\rmfamily}

%%%%%%%%%%%%%%%%%%%%%%%%%%%%%%%%%%%%%%%%%%%%%%%%%%%%%%%%%%%%%%%%%%%%%%%
% PROGRAMMCODE UND LISTINGS
%%%%%%%%%%%%%%%%%%%%%%%%%%%%%%%%%%%%%%%%%%%%%%%%%%%%%%%%%%%%%%%%%%%%%%%
\usepackage{listings}
\usepackage{xcolor}

% Grundlegende Farben für Listings
\definecolor{hellgelb}{rgb}{1,1,0.9}
\definecolor{colKeys}{rgb}{0.8,0,0.5}
\definecolor{colIdentifier}{rgb}{0.6,0,0.3}
\definecolor{colComments}{rgb}{0,0.5,0}
\definecolor{colString}{rgb}{0,0,1}

% Listing-Standardeinstellungen
\lstset{
    float=htbp,
    basicstyle=\ttfamily\color{black}\small\smaller,
    identifierstyle=,%\color{colIdentifier},
    keywordstyle=\color{colKeys}\bfseries,
    stringstyle=\color{colString},
    commentstyle=\color{colComments},
    columns=flexible,
    tabsize=4,
    frame=single,
    extendedchars=true,
    showspaces=false,
    showstringspaces=false,
    numbers=left,
    numberstyle=\tiny,
    breaklines=true,
    backgroundcolor=\color{hellgelb},
    breakautoindent=true,
    escapeinside={(*}{*)},
    literate={Ö}{{\"O}}1 {Ä}{{\"A}}1 {Ü}{{\"U}}1 {ß}{{\ss}}2 {ü}{{\"u}}1 {ä}{{\"a}}1 {ö}{{\"o}}1 {µ}{\textmu}1
}

% Sprachdefinitionen für Listings
% CSS
\lstdefinelanguage{CSS}{
    morestring=[b]',
    morestring=[b]",
    comment=[l]{/*}{*/},
    sensitive=false,
    morekeywords={accelerator,adjust,after,align,attachment,azimuth,background,before,behavior,binding,border,bottom,bottomright,bottomleft,break,caption-side,char,clear,clip,color,colors,cue,cursor,collapse,decoration,direction,display,elevation,empty-cells,family,filter,float,flow,focus,font,grid,height,image,increment,indent,input,inside,ime-mode,include-source,justify,last,layer,left,letter,line,list,margin,marker,marks,max,min,mode,modify,moz,offsett,opacity,orphans,outline,overflow,overflow-X,overflow-Y,overhang,position,position-x,position-y,padding,page,pause,pitch,play-during,position,quotes,radius,range,repeat,replace,reset,richness,right,ruby,set-link-source,select,shadow,size,spacing,speak,speak-header,speak-numeral,speak-punctuation,speech-rate,stress,stretch,style,table-layout,text,transform,text-autospace,text-kashida-space,top,topleft,topright,type,underline,unicode-bidi,use-link-source,user,variant,vertical,visibility,voice-family,volume,white-space,weight,widows,width,word,wrap,word-wrap,writing-mode,z-index,zoom}
}

% JavaScript
\lstdefinelanguage{JavaScript}{
    keywords={typeof, new, true, false, catch, function, return, null, catch, switch, var, if, in, while, do, else, case, break},
    keywordstyle=\color{colKeys}\bfseries,
    ndkeywords={class, export, boolean, throw, implements, import, this},
    ndkeywordstyle=\color{darkgray}\bfseries,
    sensitive=false,
    comment=[l]{//},
    morecomment=[s]{/*}{*/},
    morestring=[b]',
    morestring=[b]"
}

% TypoScript
\lstdefinelanguage{TypoScript}{
    keywords=[1]{PAGE, HTML, TEXT, COA, COA\_INT, FILE, IMAGE, IMG\_RESOURCE, CLEARGIF, CONTENT, RECORDS, HMENU, CTABLE, OTABLE, COLUMNS, HRULER, IMGTEXT, CASE, LOAD\_REGISTER, RESTORE\_REGISTER, FORM, SEARCHRESULT, USER, USER\_INT, TEMPLATE, FLUIDTEMPLATE, MULTIMEDIA, SVG, EDITPANEL, GIFBUILDER, GMENU, TMENU, TMENUITEM, IMGMENU, IMGMENUITEM, JSMENU, JSMENUITEM, BOX},
    keywordstyle=[1]{\color{blue}\bfseries},
    keywords=[2]{plugin, view, page, file, text, config},
    keywordstyle=[2]{\color{blue}\bfseries},
    keywords=[3]{EXT},
    keywordstyle=[3]{\color{blue}\bfseries},
    sensitive=true,
    comment=[l]{\#\ }
}

% PHP5
\lstdefinelanguage[5]{PHP}[]{PHP} {
    morekeywords={
        class,static,private,public,abstract,interface,const,function,require_once,final,new,extends,implements,
        array_combine,array_diff_uassoc,array_udiff,array_udiff_assoc,
        array_udiff_uassoc,array_walk_recursive,array_uintersect_assoc,
        array_usintersect_uassoc,array_uintersect,
        str_split, strpbrk,substr_compare,
        idate,date_sunset,date_sunrise,time_nanosleep,
        return
    }
}

% Fluid
\lstdefinelanguage[Fluid]{HTML}[]{HTML} {
    morekeywords={if, for, each, as, condition, controller, arguments, image, link, action, class}
}

%%%%%%%%%%%%%%%%%%%%%%%%%%%%%%%%%%%%%%%%%%%%%%%%%%%%%%%%%%%%%%%%%%%%%%%
% ZUSÄTZLICHE FUNKTIONALITÄT
%%%%%%%%%%%%%%%%%%%%%%%%%%%%%%%%%%%%%%%%%%%%%%%%%%%%%%%%%%%%%%%%%%%%%%%
% PDF-Einbindung
\usepackage[final]{pdfpages}

% Textformatierung
\usepackage{ragged2e}
\usepackage{lscape}

% Hilfspakete
\usepackage{ifthen}
\usepackage{forloop}
\usepackage{todonotes}
\usepackage{xspace}
\usepackage{datatool}
% !TEX encoding = UTF-8 Unicode
% !TEX root =  Main.tex

% Zeilenabstand 1,5 Zeilen ---------------------------------------------------------------------
\onehalfspacing{}


% Seitenränder ------------------------------------------------------------------------------------
\setlength{\topskip}{\ht\strutbox} % behebt Warnung von geometry
\geometry{paper=a4paper,left=21mm,right=30mm,top=30mm,bottom=20mm}


% Kopf- und Fußzeilen --------------------------------------------------------------------------
\pagestyle{scrheadings}
\renewcommand*{\chapterpagestyle}{scrheadings} 

%% Kopfzeile 
\ihead{} %linke Seite
\chead{} %mittlere Seite
\ohead{\pagemark} %rechte Seite
% \KOMAoptions{headsepline=0.4pt:0.98\paperwidth} % Trennlinie unter Kopfzeile


% Fußzeile
\ifoot{}
\cfoot{} %mittlere Seite
\ofoot{} %rechte Seite
\renewcommand{\footnotesize}{\fontsize{10pt}{12pt}\selectfont} % Fußnoten in 10pt


% sonstige typographische Einstellungen ---------------------------------------------------
\frenchspacing{}
\clubpenalty= 10000
\widowpenalty= 10000 
\displaywidowpenalty= 10000

% Quellcode-Ausgabe formatieren
\lstset{numbers=left, numberstyle=\tiny, numbersep=5pt, breaklines=true}
\lstset{emph={square}, emphstyle=\color{red}, emph={[2]root,base}, emphstyle={[2]\color{blue}}}

% Fußnoten fortlaufend durchnummerieren
\counterwithout{footnote}{chapter}

% Abstand nach Überschriften
\RedeclareSectionCommand[beforeskip=10pt, afterskip=10pt]{chapter}
\RedeclareSectionCommand[beforeskip=10pt, afterskip=5pt]{section}

% Abstand zwischen Kopfzeile und Text
\setlength{\headsep}{15pt} % Standard ist oft 20pt, kleiner machen für weniger Abstand



\begin{document}
\include{Preamble/Silbentrennung}
% !TEX encoding = UTF-8 Unicode
% !TEX root =  Main.tex
% Eigene Befehle und typographische Auszeichnungen für diese


% einfaches Wechseln der Schrift, z.B.: \changefont{cmss}{sbc}{n} ---------------------------------------
\newcommand{\changefont}[3]{\fontfamily{#1} \fontseries{#2} \fontshape{#3} \selectfont}


% Abkürzungen mit korrektem Leerraum --------------------------------------------------------------------
\newcommand{\ua}{\mbox{u.\,a.\ }}
\newcommand{\zB}{\mbox{z.\,B.\ }}
\newcommand{\dahe}{\mbox{d.\,h.\ }}
\newcommand{\Vgl}{Vgl.\ }
\newcommand{\bzw}{bzw.\ }
\newcommand{\evtl}{evtl.\ }

%Referenziert eine Abbildung mit Label 
\newcommand{\Abbildung}[1]{Abbildung~\ref{fig:#1}}

\newcommand{\bs}{$\backslash$}


% Hilfsbefehle für Standardwerte
\newcommand{\maybeAuthor}[1]{%
  \ifblank{\citeauthor{#1}}{o.\,A.}{\citeauthor{#1}}%
}
\newcommand{\maybeYear}[1]{%
  \ifblank{\citeyear{#1}}{o.\,J.}{\citeyear{#1}}%
}

% Zitat: [Meier 2020, 42] oder [o.A. o.J., 42] oder [Meier 2020]
\newcommand{\ZitatSeite}[2]{%
  [\maybeAuthor{#1}~\maybeYear{#1}%
  \ifstrempty{#2}{}{, #2}%
  ]%
}

% Zitat mit Vgl.: [Vgl. Meier 2020, S. 43] oder [Vgl. o.A. o.J., S. 43]
\newcommand{\VglZitatSeite}[2]{%
  [Vgl.\ \maybeAuthor{#1}~\maybeYear{#1}%
  \ifstrempty{#2}{}{, S.~#2}%
  ]%
}


% zum Ausgeben von Autoren
\newcommand{\AutorName}[1]{\textsc{#1}}
\newcommand{\Autor}[1]{\AutorName{\citeauthor{#1}}}


% verschiedene Befehle um Wörter semantisch auszuzeichnen -----------------------------------------------
\newcommand{\NeuerBegriff}[1]{\textbf{#1}}

\newcommand{\Fachbegriff}[2][\empty]{\ifthenelse{\equal{#1}{\empty}}{\textit{#2}\xspace}{\textit{#2}\xspace\footnote{#1}\nomenclature{#2}{#1}}}

\newcommand{\FachbegriffSpezialA}[4]{\textit{#4}\footnote{#3}\label{fn:#1}\nomenclature{#4}{#2. Siehe auch Fu{\ss}zeile auf Seite~\pageref{fn:#1}.}}

\newcommand{\FachbegriffSpezialB}[5]{\textit{#5}\footnote{#3}\label{fn:#1}\nomenclature{#4}{#2. Siehe auch Fu{\ss}zeile auf Seite~\pageref{fn:#1}.}}


% Beträge mit Währung, z.B.: \Betrag[euro]{100} oder \Betrag[dollar]{50} -----------------------------------------------------------------------------------
\newcommand{\Betrag}[2][general]{#2\,\ifthenelse{\equal{#1}{dollar}}{\$}{}\ifthenelse{\equal{#1}{euro}}{€}{}\ifthenelse{\equal{#1}{yen}}{¥}{}\ifthenelse{\equal{#1}{cent}}{¢}{}\ifthenelse{\equal{#1}{pound}}{£}{}\ifthenelse{\equal{#1}{peso}}{₱}{}\ifthenelse{\equal{#1}{baht}}{฿}{}\ifthenelse{\equal{#1}{franc}}{₣}{}\ifthenelse{\equal{#1}{lira}}{₤}{}\ifthenelse{\equal{#1}{drachma}}{₯}{}\ifthenelse{\equal{#1}{pfennig}}{₰}{}\ifthenelse{\equal{#1}{general}}{¤}{}}


% Sonstiges, z.B.: \Eingabe{ls -l}; \Code{for i in range(10):}; \Datei{main.py} ---------------------------------------------------------------------------------------------
\newcommand{\Eingabe}[1]{\texttt{#1}}
\newcommand{\Code}[1]{\texttt{#1}}
\newcommand{\Datei}[1]{\texttt{#1}}

\newcommand{\Datentyp}[1]{\textsf{#1}}
\newcommand{\XMLElement}[1]{\textsf{#1}}
\newcommand{\Webservice}[1]{\textsf{#1}}


% Beschriftung von Tabellen und Bildern ändern ----------------------------------------------------------
\addto\captionsngerman{
	\renewcommand{\figurename}{Abb.}
	\renewcommand{\tablename}{Tab.}
}


% Spaltendefinition rechtsbündig mit definierter Breite -------------------------------------------------
\newcolumntype{w}[1]{>{\raggedleft\hspace{0pt}}p{#1}}


% Linksbündige Tabellenspalten mit tabularx -------------------------------------------------------------
\newcolumntype{y}[1]{>{\RaggedRight\arraybackslash\hsize=#1\hsize}X}



\setcounter{secnumdepth}{3}
\setcounter{tocdepth}{2}


% Deckblatt und Abstract ohne Seitenzahl ---------------------------------------------------------------
% !TEX encoding = UTF-8 Unicode
% !TEX root =  Main.tex

% Kopfbereich mit Logo
\titlehead{%
  % Verschiebt das Logo nach oben
  \vspace*{-1,5cm}%
  % Zentriert das Logo
  \centering%
  % Fügt das HWR-Logo ein (60% der Textbreite)
  \includegraphics[width=0.6\linewidth]{Bilder/Logo_HWR.png}%
  % Abstand nach dem Logo
  \vspace{1,5cm}%  % Weniger Abstand nach dem Logo
}

% Art der Arbeit als Überschrift
\vspace{10cm} 
\subject{\art}

% Erstellt den Titelbereich mit Rahmen
\title{%
  \vspace*{-1,5cm}%  % Verschiebt den gesamten Titelbereich nach oben
  \normalfont\endgraf
  \rule{\textwidth}{.4pt}\\\relax
  \vspace{0.5cm}%  % Kleiner Abstand nach dem Zeilenumbruch
  \begingroup
    \centering
    % Zeilenabstand auf 1.5
    \linespread{1.5}\selectfont
    % Titel in großer Schrift
    \large\titel\\
    % Untertitel in normaler Schrift (falls vorhanden)
    % \normalsize\untertitel
  \endgroup%  % Weniger Abstand vor der unteren Linie
  \endgraf\rule{\textwidth}{.4pt}%
}

% Setzt das Datum und die Hochschule
\date{\vspace{-1.5cm}%  % Reduziert den Abstand zum Titelbereich
\normalsize vorgelegt am \datum\\
an der \\ 
Hochschule für Wirtschaft und Recht Berlin\vspace{5mm}}

% Erstellt die Tabelle mit den Informationen
\publishers{%
  \begin{tabular}{l l}
    % Persönliche Informationen
    \textbf{\normalsize{Von: }} & \normalsize{\autor} \\
    \textbf{\normalsize{Fachrichtung:}} & \normalsize{\fachrichtung} \\
    \textbf{\normalsize{Studienjahrgang:}} & \normalsize{\studienjahrgang} \\
    \textbf{\normalsize{Semester:}} & \normalsize{\semester} \\
    \textbf{\normalsize{Ausbildungsbetrieb:}} & \normalsize{\ausbildungsbetrieb} \\
    % Betreuer-Informationen
    \textbf{\normalsize{Betreuender Prüfer:}} & \normalsize{\erstgutachter} \\
    \textbf{\normalsize{Betreuer Betrieb:}} & \normalsize{\zweitgutachter} \\
    % Unterschriftszeile
    \textbf{\normalsize{Unterschrift Betreuer: }} & \underline{\hspace{3cm}} \\
  \end{tabular}
}

% Keine Seitenzahl auf dem Deckblatt
\pagestyle{empty}
% Erstellt das Deckblatt
\maketitle

% Seitennummerierung -----------------------------------------------------------------------------------
%   Vor dem Hauptteil werden die Seiten in großen römischen Ziffern 
%   nummeriert.
% ------------------------------------------------------------------------------------------------------
\pagenumbering{Roman}
\tableofcontents{}



% Abkürzungsverzeichnis --------------------------------------------------------------------------------
% !TEX encoding = UTF-8 Unicode
% !TEX root =  ../Bachelorarbeit.tex
\chapter*{Abkürzungsverzeichnis}
\markboth{Abkürzungsverzeichnis}{Abkürzungsverzeichnis}
\addcontentsline{toc}{chapter}{Abkürzungsverzeichnis}

\printglossary[type=\acronymtype, title=Abkürzungsverzeichnis]


\newacronym{rup}{RUP}{Rational Unified Process}
\nomenclature{UKW}{Ultrakurzwelle -- Synonym f\"ur UKW-Rundfunk im Bereich 87,5 bis 108 MHz des VHF-Bandes II}

% arabische Seitenzahlen im Hauptteil ------------------------------------------------------------------
\clearpage{}
\pagenumbering{arabic}

% die Inhaltskapitel werden in "Inhalt.tex" inkludiert -------------------------------------------------
% !TEX encoding = UTF-8 Unicode
% !TEX root =  Main.tex

% Hier können die einzelnen Kapitel inkludiert werden. Sie müssen in den 
% entsprechenden .TEX-Dateien vorliegen. Die Dateinamen können natürlich 
% angepasst werden.

% !TEX encoding = UTF-8 Unicode
% !TEX root =  ../Main.tex

\chapter{Einleitung}
\label{cha:Einleitung}

\include{Inhalt/02_Methodik}
\include{Inhalt/03_Theoretischer_Rahmen}
\include{Inhalt/04_Analyse_und_Design}
\include{Inhalt/05_Realisierung}
\include{Inhalt/06_Fazit}


% Literaturverzeichnis ---------------------------------------------------------------------------------
%   Das Literaturverzeichnis wird aus der BibTeX-Datenbank "Bibliographie.bib"
%   erstellt.
% ------------------------------------------------------------------------------------------------------
\bibliography{Literatur/Bibliographie} % Aufruf: bibtex Masterarbeit
\bibliographystyle{natdin} % DIN-Stil des Literaturverzeichnisses


% Restliche Verzeichnisse -> wenn diese leer sind, sollte das hier auskommentiert werden ------------------------------------------------------------------------------
\listoffigures{} % Abbildungsverzeichnis
\listoftables{} % Tabellenverzeichnis


% Ehrenwörtliche Erklärung ----------------------------------------------------------------------------
% !TEX encoding = UTF-8 Unicode
% !TEX root =  Main.tex

%muss nicht mehr geändert werden, ist schon nach HWR Vorlage

\addchap{Ehrenwörtliche Erklärung}
Ich erkläre ehrenwörtlich:

dass ich die vorliegende \art\xspace in allen Teilen selbstständig angefertigt und keine anderen als die in der Arbeit angegebenen Quellen und Hilfsmittel benutzt habe, und dass die Arbeit in gleicher oder ähnlicher Form in noch keiner anderen Prüfung vorgelegen hat. Sämtliche wörtlichen oder sinngemäßen Übernahmen und Zitate, sowie alle Abschnitte, die mithilfe von Kl- basierten Tools entworfen, verfasst und/oder bearbeitet wurden, sind kenntlich gemacht und nachgewiesen. Im Anhang meiner Arbeit habe ich sämtliche KI-basierte Hilfsmittel angegeben. Diese sind mit Produktnamen und formulierten Eingaben (Prompts) in einem Kl-Verzeichnis ausgewiesen.

Ich bin mir bewusst, dass die Verwendung von Texten oder anderen Inhalten und Produk- ten, die durch Kl-basierte Tools generiert wurden, keine Garantie für deren Qualität darstellt. Ich verantworte die Ubernahme jeglicher von mir verwendeter maschinell generierter Passagen vollumfänglich selbst und trage die Verantwortung für eventuell durch die Kl generierte fehlerhafte oder verzerrte Inhalte, fehlerhafte Referenzen, Verstöße gegen das Datenschutz- und Urheberrecht oder Plagiate.\\

\ort, den \today

% hier kann eine Unterschrift eingefügt werden, wenn die Arbeit ausgedruckt wird
% \begin{flushright}
%     \includegraphics[height=2cm]{Bilder/Unterschrift.png}
% \end{flushright}


\autor\\
 


% Anhang -----------------------------------------------------------------------------------------------
%   Die Inhalte des Anhangs werden analog zu den Kapiteln inkludiert.
%   Dies geschieht in der Datei "Anhang.tex".
% ------------------------------------------------------------------------------------------------------
\appendix
    % \clearpage{}
    % \pagenumbering{roman}
    \chapter{Anhang}
    \label{sec:Anhang}
    % Rand der Aufzählungen in Tabellen anpassen
    \setdefaultleftmargin{1em}{}{}{}{}{}
    % !TEX encoding = UTF-8 Unicode
% !TEX root =  Main.tex

% Beispiel: 
% \section{Beiliegende CD}
% \label{sec:BeiliegendeCd}



\end{document}
